% xelatex
\documentclass{article}

\usepackage[utf8]{inputenc}
\usepackage{fontspec}
\usepackage[margin=.75in]{geometry}

\usepackage{ot-tableau}
\usepackage[backend=biber, style=authoryear-icomp]{biblatex}
\usepackage{textgreek}
\usepackage{easylist}
\usepackage{hanging}
\usepackage{hyperref}
\usepackage{blindtext}
\usepackage{tipa}
\usepackage{cgloss4e}
\usepackage{gb4e}
\usepackage{qtree}
\usepackage{enumerate}
\usepackage{longtable}
\usepackage{textgreek}
\usepackage{amsmath,amssymb,latexsym}
\usepackage{wasysym}
\addbibresource{$HOME/Documents/LaTeX/uni.bib}


\title{Bio Final Review}
\author{Danny Filipski, Chris, and others}

\begin{document}

\maketitle

\section{General Information}

For this, see the google doc.

\section{Genetics}

\subsection{Sex-linked genes}

These are genes located on the sex chromosomes.
They will show different phenotype frequencies based on gender.

\begin{exe}
\ex Gene A is on the X chromosome.
A is the wild type, and \textalpha \ is the diseased type.\\
X\textsuperscript{A}X\textsuperscript{\textalpha} $\times$ X\textsuperscript{A}Y:\\

\begin{tabular}{l | l}
X\textsuperscript{A}Y & X\textsuperscript{\textalpha}Y\\
X\textsuperscript{A}X\textsuperscript{A}  & X\textsuperscript{\textalpha}X\textsuperscript{A}\\

\end{tabular}
\end{exe}

\subsection{Pedigrees}
\Circle \ = Unaffected Female\\
\CIRCLE \ = Affected Female\\
$\square$ = Unaffected Male\\
$\blacksquare$ = Affected Male\\
Connecting lines on pedigrees work just as they do on family trees.
Relatively simple logic can be used to determine the genotypes of each member of the pedigree; however, some can be more difficult than others.
My general method is to use the ``method of staring'' in the words of Mr. Letarte.

\subsection{Genetic Disorders -- Sickle Cell, Cystic Fibrosis, Huntington's}

\subsection{Nondisjunction}

\subsection{Recombinant DNA -- Restriction Enzymes, Ligase, Electrophoresis, GFP, PCR, Selectable Markers, Screens, Plasmids, Transformations}

\subsection{Selective Breeding -- Hybridization vs Inbreeding}

<++>


\end{document}
