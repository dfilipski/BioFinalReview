% xelatex
\documentclass{article}

\usepackage[utf8]{inputenc}
\usepackage{fontspec}
\usepackage[margin=.75in]{geometry}

\usepackage{ot-tableau}
\usepackage[backend=biber, style=authoryear-icomp]{biblatex}
\usepackage{textgreek}
\usepackage{easylist}
\usepackage{hanging}
\usepackage{hyperref}
\usepackage{blindtext}
\usepackage{tipa}
\usepackage{cgloss4e}
\usepackage{gb4e}
\usepackage{qtree}
\usepackage{enumerate}
\usepackage{longtable}
\usepackage{textgreek}
\usepackage{amsmath,amssymb,latexsym}
\usepackage{wasysym}
\addbibresource{$HOME/Documents/LaTeX/uni.bib}

\pagenumbering{gobble}

\title{Bio Final Review}
\author{Daniel Filipski}

\begin{document}

\maketitle

\section{General Information}

For this, see the google doc.

\section{Genetics}

\subsection{Sex-linked genes}

These are genes located on the sex chromosomes.
They will show different phenotype frequencies based on gender.

\begin{exe}
\ex Gene A is on the X chromosome.
A is the wild type, and \textalpha \ is the diseased type.\\
X\textsuperscript{A}X\textsuperscript{\textalpha} $\times$ X\textsuperscript{A}Y:\\

\begin{tabular}{l | l}
X\textsuperscript{A}Y & X\textsuperscript{\textalpha}Y\\
X\textsuperscript{A}X\textsuperscript{A}  & X\textsuperscript{\textalpha}X\textsuperscript{A}\\

\end{tabular}
\end{exe}

\subsection{Pedigrees}
\Circle \ = Unaffected Female\\
\CIRCLE \ = Affected Female\\
$\square$ = Unaffected Male\\
$\blacksquare$ = Affected Male\\
Connecting lines on pedigrees work just as they do on family trees.
Relatively simple logic can be used to determine the genotypes of each member of the pedigree; however, some can be more difficult than others.
My general method is to use the ``method of staring'' in the words of Mr. Letarte.

\subsection{Genetic Disorders -- Sickle Cell, Cystic Fibrosis, Huntington's}
\textbf{Sickle Cell}
\begin{itemize}
\item Red blood cells contain hemoglobin, which bind O\textsubscript{2}
\item Hemoglobin is made up of two \textalpha -goblin and two \textbeta -globin polypeptides
\item Mutation in \textbeta -globin makes it slightly less soluble
\item When O\textsubscript{2} is low, hemoglobin without O\textsubscript{2} will start to clump and form long fibers that will change the shape of the red blood cell, which will then get stuck in cappilaries
\item If one is a heterozygote of this disease, they have an advantage against milleria
\item Sickle Cell disease is recessive, because its effects are not great enough with only some of the \textbeta -globin broken.
\end{itemize}

\textbf{Cystic Fibrosis}
\begin{itemize}
\item In frame three base pair deletion in gene for CFTR
\item CFTR is missing one amino acid (phenylalanene), which causes it to misfold and be destroyed
\item CFTR is a channel in the epithelial cell membranes for Cl\textsuperscript{-}
\item Without CFTR, there is too much extracellular Cl\textsuperscript{-}, which makes the fluid outside the cell thicker
\item Mucus clogs lungs and serves as a growth substance for pseudomanas aerougenase
\item The allele for Cystic Fibrosis is recessive, as with one of the two CFTR, cells still have enough paths for Cl\textsuperscript{-}
\end{itemize}

\textbf{Huntington's Disease}
\begin{itemize}
\item Mutation is dominant, but the disease does not present itself untill late 30's or early 40's
\item Huntingtin gene expressed in nerve cells. Its developemental role in adults is unclear
\item CAG (codes for glutamine)
\item Wild type 6-35 repeats
\item Diseased 36+ repeats
\item Diseased protein forms aggretes in neurons, which lead to cell death.
\end{itemize}

\subsection{Nondisjunction}

Nondisjunction Event -- Failure to separate chromosomes\\
This is more common in meiosis I. Trisomy 21 causes downsyndrome.

\subsection{Recombinant DNA -- Restriction Enzymes, Ligase, Electrophoresis, GFP, PCR, Selectable Markers, Screens, Plasmids, Transformations}
Recombinant DNA -- Combination of two or more pieces of DNA to create an artificial construct.\\
\\
Building Pieces of DNA:
\begin{enumerate}
\item Synthesize from scrath
\item Cut and Paste
\begin{itemize}
\item Ligase is not specific and will join any two pieces of DNA.
\item Restriction enzymes originate from bacteria, where they served s a type of immune system.
\item Restriction System: methylase adds CH\textsubscript{3}
\item Restriction Enzyme cuts DNA if not regulated.
\end{itemize}
\end{enumerate}

Eco RI:
\begin{itemize}
\item Sticky ends of $GAATTC$ (a DNA reverse palindrome)
\item Eco RI cuts between the G and the A
\end{itemize}

Plasmid:
\begin{itemize}
\item Mini chromosome in bacteria
\item Must have an ori, a selectable marker
\item Plasmids can be shared between cells. They also can be picked up from the enviroment, when they are there for whatever reason.
\item An example of plasmids which can be shared between cells is antibiotic resistance.
\end{itemize}

DNA Sequencing:
\begin{itemize}
\item Denature DNA into single strands
\item Add primer for only one strand
\item Provide DNA polymerase and the four nucleotides, with a small fraction of the nucleotides modified so that DNA polymerase cannot extend from them (remove 3 hydroxyl)
\end{itemize}

\subsection{Selective Breeding -- Hybridization vs Inbreeding}

Selective breeding -- only allowing parents with certain characteristics to breed.\\
Hybridization:
\begin{itemize}
\item Crossing two organisms (typically plants) to get the best traits from both in a hybrid
\item Can be different species.
\end{itemize}\
Inbreeding:
\begin{itemize}
\item Continued breeding of individuals with similar characteristics
\item Dramatically decreases genetic variation
\item Increases prominance of some traits
\end{itemize}

\section{Evolution}

\subsection{Darwin's Observations and Background Knowledge}

\subsection{Evolutionary Theory -- Adaptations, Fitness, Natural Selection, etc.}

\subsection{Evidence for Evolution}

\subsection{Population Genetics -- Hardy-Weinberg Equation and Conditions that disrupt equilibrium, selection on polygenic traits}

\subsection{Speciation}

\subsection{Molecular Clocks}

\subsection{Binomial Nomenclature and Linnaean Classification}

\subsection{Cladograms}

\subsection{Domains and Kingdoms}

\subsection{Fossil Record}

\subsection{Speciation and Extinction Events}

\subsection{Origins of Life including Endosymbiont Theory}

<++>







\end{document}
